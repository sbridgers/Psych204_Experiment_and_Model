% 
% Annual Cognitive Science Conference
% Sample LaTeX Paper -- Proceedings Format
% 

% Original : Ashwin Ram (ashwin@cc.gatech.edu)       04/01/1994
% Modified : Johanna Moore (jmoore@cs.pitt.edu)      03/17/1995
% Modified : David Noelle (noelle@ucsd.edu)          03/15/1996
% Modified : Pat Langley (langley@cs.stanford.edu)   01/26/1997
% Latex2e corrections by Ramin Charles Nakisa        01/28/1997 
% Modified : Tina Eliassi-Rad (eliassi@cs.wisc.edu)  01/31/1998
% Modified : Trisha Yannuzzi (trisha@ircs.upenn.edu) 12/28/1999 (in process)
% Modified : Mary Ellen Foster (M.E.Foster@ed.ac.uk) 12/11/2000
% Modified : Ken Forbus                              01/23/2004
% Modified : Eli M. Silk (esilk@pitt.edu)            05/24/2005
% Modified : Niels Taatgen (taatgen@cmu.edu)         10/24/2006
% Modified : David Noelle (dnoelle@ucmerced.edu)     11/19/2014

%% Change "letterpaper" in the following line to "a4paper" if you must.

\documentclass[10pt,letterpaper]{article}

\usepackage{cogsci}
\usepackage{pslatex}
\usepackage{apacite}
\usepackage{color}
\usepackage{url}
%\usepackage{hyperref}
%\hypersetup{
%    colorlinks=true,
%    linkcolor=blue,
%    filecolor=magenta,      
%    urlcolor=cyan,
%}

\definecolor{Red}{RGB}{255,0,0}
\newcommand{\red}[1]{\textcolor{Red}{#1}}  



\title{Understanding the Costs and Benefits of Learning When Teaching Others}
 
\author{{\large \bf Sophie Bridgers (sbridge@stanford.edu)} \\
  Department of Psychology, Jordan Hall, 450 Serra Mall \\
 Stanford, CA 94305 USA
  \AND {\large \bf Emily Tang (emjtang@stanford.edu)} \\
   Department of Psychology, Jordan Hall, 450 Serra Mall \\
 Stanford, CA 94305 USA}

\begin{document}

\maketitle


\begin{abstract}
To-Do

\textbf{Keywords:} 
pedagogy; social learning; cost-benefit analysis; Bayesian models
\end{abstract}


\section{Introduction}

Teaching is one of the most powerful forms of human communication, enabling a rapid, effective dissemination of knowledge and culture within and across generations. Remarkably, we decide what information to teach without direct access to the learner's mind and often without explicit requests for information. How do we make this inference and how does this capacity develop? 

Consider a naive learner who encounters a novel toy and tries to figure out how it works via self-guided exploration. The learner likely has her own ideas about how the toy works and will test these hypotheses by intervening on the toy. However, this generation of evidence is costly, requiring time and effort. Additionally, the value of this exploration is uncertain as the learner might never successfully activate the toy. Teachers can help decrease the costs of learning by communicating information that would be too difficult, time-consuming, or perhaps even impossible to obtain on one's own. Likewise, teachers can increase the value of learning by providing the critical information needed to support the most accurate inference. Thus, an effective teacher should be sensitive to opportunities for maximizing the benefits of learning (the value of the information gained) and minimizing its costs (the effort needed to acquire it). However, neither the costs nor benefits of learning are directly observable; one must consider the learner's prior knowledge and goals, and reason forward in time to calculate the expected costs and benefits of learning.



Prior research suggests that even young children might be capable of such a sophisticated inference. Both infants and children expect agents to act in accordance with the principle of rationality -- i.e., that agents will act efficiently in pursuit of their goals \cite{Baker2009}. 
Looking time studies reveal that infants anticipate an agent to choose the action requiring the least amount of effort to achieve her goal \cite{Scott2013}.
Preschool-aged children, like adults, go beyond simply reasoning about the overall utility of agents' goal-directed actions. 
Rather, they consider the relative costs and rewards of pursuing certain goals over others and make judgments consistent with a naive utility calculus, expecting agents to incur higher costs for greater rewards \cite{JaraEttinger2015}.

Relatedly, children and adults consider both the value and costs of information transfer when evaluating teachers, preferring teachers who provide enough evidence to support accurate inference over those who are exhaustively informative \cite{GweonReview, Shafto2012}. Likewise, children can decide how much information to teach given the learner's prior knowledge and goals and the costs of information transfer: If the learner has relevant prior knowledge, children are selective in how much they teach unless providing exhaustive information is no more costly (i.e., they are allowed to verbally explain how a toy works rather than needing to physically demonstrate it)  \cite{GweonSchulz2014}. Building upon this work, Bridgers and Gweon hypothesize that children (and adults) can flexibly decide what information to teach via an intuitive cost-benefit analysis. More specifically, they predict that children (5 to 7 years old) reason about the expected benefits and costs of learning on one's own (from the perspective of the learner), and successfully choose to teach information that maximizes the benefits and minimizes the costs. 

In a current experiment, Bridgers and Gweon are investigating whether children selectively teach information that decreases the expected costs of learning. In this study, children explore two novel toys. One toy has an obvious causal mechanism that is easy to discover and that produces an exciting effect (a single, red button that causes the toy to light-up and spin). The other toy has a non-obvious mechanism that is difficult to discover without instruction and that produces a relatively less interesting effect (two of seven buttons must be pressed simultaneously to make the toy play music). Once children learn how the toys work, they are asked to choose one to teach to a naive learner who will play with the toys later all by herself. Given children's sensitivity to differences in the costs of verbal communication and additional actions on a toy, Bridgers and Gweon predict children will reason about the expected costs of figuring out the toys on one's own vs. from instruction and will more often teach the difficult toy than the easy toy, but not in the control condition where each toy is equally easy to figure out.

Here, we formalize the intuitions Bridgers and Gweon investigate with a computational theory of efficient teaching. We develop a Bayesian cognitive model of a knowledgeable teacher reasoning about a naive learner who wants to figure out how two toys work. In this simplified scenario, the teacher selects one of two toys to teach the learner and the learner discovers how the other toy works on her own. We assume the teacher is helpful and that her goal is to teach the toy that will maximize the probability that the learner figures out how to activate both toys. The teacher, thus, critically bases her decision on the relative costs for the learner of figuring out how the toys work without instruction. 

In this paper, we first describe the details of the model and show that it predicts the teacher will teach the toy that has a higher cost of learning but when costs are equal, the toy with the higher value. Second, we compare the model's predictions to adult performance in an experiment based on Bridgers and Gweon's experiments with children. Finally, we conclude by discussing how the predictions of the model help us to better understand how we decide what to teach and how the model could be extended to incorporate the teacher's reasoning about her own costs and benefits of teaching to better account for the dyadic nature of pedagogical interactions.

\section{Model}

Central to our modeling approach are the ideas discussed in chapter 6 (Inference about inference) of Probabilistic Models of Cognition \cite{Goodman}. Our efficient teaching model merges the vending machine model and the teacher-learner model (both presented in chapter 6).  We modified the 'make-vending-machine' function to a 'make-toy' function: toys take a single action as an argument and return either an effect or nothing. Each possible action on a toy has a certain probability of producing the effect. Put simply, each toy has one to multiple buttons, and when pressed these buttons produce an outcome (e.g., something lights up) with high probability (90\%) or low probability (10\%). In the scenarios we consider, there are always two toys, which we will refer to as toy A and toy B, varying in learning cost and value (e.g., a toy that is easy to learn but high in value versus a toy that is hard to learn but low in value). The cost of learning is represented as the uncertainty over what action will cause the toy to produce an outcome; thus, cost increases as the number of buttons (possible actions) increases. The value of learning is represented as the quantity of the outcome (currently the number of lights). For all of the toys we consider, one button produces the effect(s) with high probability and all additional buttons with low probability.

The learner is represented as a function that takes in her goal (to activate both toys) and a message from the teacher about how one of the two toys works. Prior to receiving the teacher's message, the learner does not know how either of the toys works (expressed as not knowing the probability of effect associated with the possible actions on each toy).  As a result of the teacher's message, the learner knows how one toy works but does not know about the other toy and will need to figure it out on her own. We query the learner model for the actions she takes on toy A and toy B conditioning on her goal being satisfied (i.e., that she successfully activates the toys). The learner function thus returns the probability that the leaner will successfully activate both toys, $a$, given the message the teacher sends about one toy, $t$, and the learner's goal of activating both, $g_{learner}$ ($P_{learner}(a | t, g_{learner})$). 

The teacher is represented as a function with no arguments. The teacher knows how both toy A and toy B works (expressed as having knowledge of the probability of effect for each possible action on the toys). The teacher also has knowledge of the learner's goal to activate both toys. The learner's goal is represented within the teacher-model as a boolean utility function that takes the outcomes of the learner's actions on each toy as arguments. The function returns true if the learner's actions bring about an effect and false otherwise. We assume that the learner wants to maximize the reward of her actions and see the highest possible number of effects (e.g., the highest number of lights); however, we also assume that the learner is \textit{soft-max} optimal. We implement this notion in the model by making the conditions that satisfy the learner's goal probabilistic: the learner's goal function is most likely to return true if the learner's actions bring about the maximal effect across the two toys, but sometimes it returns true for sub-optimal returns. 

We assume the teacher is helpful and that her goal is for the learner to figure out how both toys work (i.e., that the learner will perform an action on each toy that successfully brings about an effect). The teacher's goal is also represented as a boolean function that takes the learner's actions on the toys as arguments and returns true (i.e, the goal is satisfied) if these actions bring about an effect and false otherwise. Similar to our assumptions about the learner, we assume the teacher is \textit{soft-max} optimal and implement this notion in the same way we do for the learner. 

However, though the teacher wants the learner to successfully activate both toys, the teacher can only teach the learner how \textit{one} of the toys works. We formalize this constraint as the teacher sends the button-effects for one toy and "naive-notions" of the button effects for the other toy (i.e., that each button on this toy is equally likely to bring about the effect). Thus, the teacher must decide to teach the toy that will maximize the probability that the learner will successfully activate both toys; we expect the teacher to select the toy that reduces the greatest amount of uncertainty for the learner about how the toys work. We query the teacher model for the toy she teaches (i.e., the message with the button-effects for \textit{either} toy A or toy B that she sends to the learner-model), and we condition on her goal being satisfied. The teacher function thus returns the probability that the teacher will teach one of the toys, $t$, given the teacher's goal, $g_{teacher}$, defined in terms of the effects the learner's actions (based on the teacher's message) achieve ($P_{teacher}(t | g_{teacher})$).

For complete details on how we implement our model of efficient teaching, please see the \url{https://github.com/sbridgers/Psych204_Experiment_and_Model/blob/master/ModelCode/TeacherModel.md} for the model online.

\subsection{Model Predictions}

We test the predictions of our model for a series of scenarios comparing toys varying in expected learning cost and value. We find that for each of these scenarios, the model's predictions are consistent with our intuitions. 

We first see what the model predicts the teacher will teach when choosing between two toys equal in cost but unequal in value. We find, as we would expect, that the model does not make a strong prediction in this scenario: the probability that the teacher teaches each of the toys is approximately 50\%. Next, we see what the model predicts when toy A and toy B are equal in cost (e.g., they each have two buttons) but unequal in value (e.g., toy A produces two effects, while toy B only produces one). We find that in this scenario, the teacher teaches toy A with a higher probability than toy B ($P(toyA) \sim 57\%$). 

When choosing between two toys unequal in cost but equal in value, the model predicts the teacher will teach the toy with the higher expected cost of learning. For example, if both toys produce two effects but toy A has one button, while toy B has 7, the model predicts the teacher will teach toy B with higher probability than toy A ($P(toyB) \sim 60\%$). Relatedly, if toy B is associated with the higher cost \textit{and} the higher value (e.g., toy B has 7 buttons and produces two effects, while toy A has one button and only produces one effect), the model again predicts the teacher will teach toy B with higher probability than toy A ($P(toyB) \sim 65\%$).

Thus far the scenarios we have considered have fairly straight forward intuitive predictions, and we have shown that the model's predictions match our own. When the toys are matched in cost and value, the model predicts that teacher will teach each toy with equal probability. When the toys are matched in value, but unequal in cost, the model predicts that the teacher will teach the toy with higher cost with greater probability than the toy with lower cost. When the toys are mis-matched in value and unequal in cost, the model predicts that the teacher will most probably teach the toy with higher cost when that toy also has the higher value. But what about when the toy with higher cost is lower in value than the toy with lower cost? What toy should the teacher teach: the toy that has a higher expected value of learning or the toy that has a higher expected cost? We assume that in this scenario teaching the toy with higher cost is the most helpful because it reduces the greatest amount of uncertainty for the learner and maximizes the probability that she will be able to successfully activate \textit{both} toys. We find that the model's predictions here are again consistent with our intuitions: when toy B is higher in cost but lower in value than toy A, the model predicts the teacher will teach toy B with a higher probability than toy A ($P(toyB) \sim 55\%$).

\section{Experiment 1: Testing Adult Intuitions}

We conduct an experiment with adults to validate the predictions of our model. The experiment is based on Bridgers and Gweon's experiments with children. Adults were presented with pairs of toys varying in relative cost and value (as described in each of the scenarios considered in the modeling section above). After exploring the toys and figuring out how they work, adults were asked to choose one of the toys to teach to a child learner who wanted to figure out how both toys work. 

\subsection{Method}

\subsubsection{Participants}

To-Do: Adults will be tested via an online survey on Amazon Mechanical Turk. Participation will be restricted to Amazon Mechanical Turk workers who have a HIT acceptance rate of 85\% and U.S. IP addresses. We will aim to have balanced numbers of male and female participants.

\subsubsection{Stimuli}

To-Do: We are still finalizing the stimuli. They will be based on the stimuli used by Bridgers and Gweon in their current experiments with children.

\subsubsection{Procedure}

You can view the current draft of our experiment \url{http://web.stanford.edu/~sbridge/experiments/teaching-toys/Experiment/experiment/toyExperiment.html}.

\subsection{Results}

To-Do

\subsection{Discussion}

To-Do

\section{Conclusion}

To-Do

\bibliographystyle{apacite}

\setlength{\bibleftmargin}{.125in}
\setlength{\bibindent}{-\bibleftmargin}

\bibliography{cogsci_template}


\end{document}
